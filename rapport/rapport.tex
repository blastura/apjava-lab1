\documentclass[a4paper, 12pt]{article}
\usepackage[swedish]{babel}
\usepackage[utf8]{inputenc}
\usepackage{verbatim}
\usepackage{fancyhdr}
\usepackage{graphicx}
\usepackage{parskip}
% Include pdf with multiple pages ex \includepdf[pages=-, nup=2x2]{filename.pdf}
\usepackage[final]{pdfpages}
% Place figures where they should be
\usepackage{float}

% vars
\def\title{Plugin 1.2}
\def\preTitle{Laboration 1}
\def\kurs{Applikationsprogrammering i Java, HT-08}

\def\namn{Anton Johansson}
\def\mail{dit06ajn@cs.umu.se}
\def\pathToCode{$\sim$dit06ajn/edu/apjava/lab1}

\def\handledareEtt{Johan Eliasson johane@cs.umu.se}
\def\inst{datavetenskap}
\def\dokumentTyp{Laborationsrapport}

\begin{document}
\begin{titlepage}
  \thispagestyle{empty}
  \begin{small}
    \begin{tabular}{@{}p{\textwidth}@{}}
      UMEÅ UNIVERSITET \hfill \today \\
      Institutionen för \inst \\
      \dokumentTyp \\
    \end{tabular}
  \end{small}
  \vspace{10mm}
  \begin{center}
    \LARGE{\preTitle} \\
    \huge{\textbf{\kurs}} \\
    \vspace{10mm}
    \LARGE{\title} \\
    \vspace{15mm}
    \begin{large}
        \namn, \mail \\
        \pathToCode
    \end{large}
    \vfill
    \large{\textbf{Handledare}}\\
    \mbox{\large{\handledareEtt}}
  \end{center}
\end{titlepage}

\pagestyle{fancy}
\rhead{\today}
\lhead{\namn, \mail}
\chead{}
\lfoot{}
\cfoot{}
\rfoot{}

\tableofcontents
\newpage

\rfoot{\thepage}
\pagenumbering{arabic}

\section{Problemspecifikation}
% Beskriv med egna ord vad uppgiften gick ut på. Är det någonting som
% varit oklart och ni gjort egna tolkningar så beskriv dessa.
Denna laboration gick ut på att skriva ett Javaprogram som använder
sig av Javas paket \textit{java.lang.reflect}. Paketet har funktioner
som gör det möjligt att inspektera, instansera, modifiera och exekvera
objekt och klasser medan ett program körs.

Till detta program skulle ett grafiskt gränssnitt skrivas. Från detta
grafiska gränssnitt kan metoder exekveras och eventuell utdata skrivs
ut på ett textfält.

Problemspecifikation finns i original på:\\
\verb!http://www.cs.umu.se/kurser/5DV085/HT08/labbar/lab1.html!

\section{Användarhandledning}
% Förklara var programmet och källkoden ligger samt hur man startar,
% kompilerar och använder det.

\section{Systembeskrivning}
% Beskriv översiktligt hur programmet är uppbyggt och hur det löser
% problemet.

\section{Begränsningar}
% Vilka problem och begränsningar har din lösning av uppgiften? Hur
% skulle de kunna rättas till?

% TODO - Eftersom man skickar in klassen när man kör programmet från
% terminalen så sköts alla undantag i main metoden. Det erbjuds ingen möjlighet
% att ändra klass i det grafiska gränssnittet. Hade det grafiska
% gränssnittet kunnat hantera detta hade dessa try catch block
% flyttats upp bland den koden.

% TODO - Vad händer om man laddar en klass utan metoder, inga
% relevanta felmeddelanden och det kan bli nullpointer

\section{Reflektioner}
% Var det något som var speciellt krångligt? Vilka problem uppstod och
% hur löste ni dem? Allmänna synpunkter. Hur skulle man kunna använda
% dessa metoder i andra mer omfattande system?

\section{Testkörningar}
%Noggranna testkörningar där man ser att programmet fungerar som det ska.

\section{Diskussion}
% Hur fungerade det att följa en kodkonvention? Vilka var fördelarna
% respektive nackdelarna?

\section{Källkod}
% Källkoden ska finnas tillgänglig i er hemkatalog
% ~/edu/apjava/lab1/. Bifoga även utskriven källkod. Rekommenderat är
% att ni skriver ut den med kommandot
\end{document}
